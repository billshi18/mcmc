
\documentclass[11pt]{article}

\usepackage{amsmath}
\usepackage{amsfonts}
\usepackage{indentfirst}
\usepackage{natbib}
\usepackage{url}
\usepackage[utf8]{inputenc}

\newcommand{\real}{\mathbb{R}}

\DeclareMathOperator{\prior}{pri}
\DeclareMathOperator{\posterior}{post}
\DeclareMathOperator{\indicator}{ind}

\newcommand{\fatdot}{\,\cdot\,}

% \VignetteIndexEntry{Bayes Factors via Serial Tempering}

\usepackage{/APPS/lib64/R/share/texmf/Sweave}
\begin{document}

\title{Bayes Factors via Serial Tempering}
\author{Charles J. Geyer}
\maketitle


\section{Introduction}

\subsection{Bayes Factors} \label{sec:bayes-factors}

Let $\mathcal{M}$ be a finite or countable set of models (here we only
deal with finite $\mathcal{M}$ but Bayes factors make sense for countable
$\mathcal{M}$).  For each model
$m \in \mathcal{M}$ we have the prior probability of the model $\prior(m)$.
It does not matter if this prior on models is unnormalized.

Each model $m$ has a parameter space $\Theta_m$ and a prior
$$
   g(\theta \mid m), \qquad \theta \in \Theta_m
$$
The spaces $\Theta_m$ can and usually do have different dimensions.  That's
the point.  These within model priors must be normalized proper priors.
The calculations to follow make no sense if these priors are unnormalized
or improper.

Each model $m$ has a data distribution
$$
   f(y \mid \theta, m)
$$
and the observed data $y$ may be either discrete or continuous
(it makes no difference to
the Bayesian who treats $y$ as fixed after it is observed and treats
only $\theta$ and $m$ as random).

The unnormalized posterior for everything
(for models and parameters within models)
is
$$
   f(y \mid \theta, m) g(\theta \mid m) \prior(m)
$$
To obtain the conditional distribution of $y$ given $m$, we must integrate
out the nuisance parameter $\theta$
\begin{align*}
   q(y \mid m)
   & =
   \int_{\Theta_m} f(y \mid \theta, m) g(\theta \mid m) \prior(m) \, d \theta
   \\
   & =
   \prior(m) \int_{\Theta_m} f(y \mid \theta, m) g(\theta \mid m) \, d \theta
\end{align*}
These are the unnormalized posterior probabilities of the models.  The
normalized posterior probabilities are
$$
   \posterior(m \mid y)
   =
   \frac{ q(y \mid m) }{ \sum_{m \in \mathcal{M}} q(y \mid m) }
$$

It is considered useful to define
$$
   b(y \mid m)
   =
   \int_{\Theta_m} f(y \mid \theta, m) g(\theta \mid m) \, d \theta
$$
so
$$
   q(y \mid m) = b(y \mid m) \prior(m)
$$
Then the ratio of posterior probabilities of models $m_1$ and $m_2$ is
$$
   \frac{\posterior(m_1 \mid y)}{\posterior(m_2 \mid y)}
   =
   \frac{q(y \mid m_1)}{q(y \mid m_2)}
   =
   \frac{b(y \mid m_1)}{b(y \mid m_2)}
   \cdot
   \frac{\prior(m_1)}{\prior(m_2)}
$$
This ratio is called the \emph{posterior odds} of the models (a ratio of
probabilities is called an \emph{odds}) of these models.

The \emph{prior odds} is
$$
   \frac{\prior(m_1)}{\prior(m_2)}
$$

The term we have not yet named in
$$
   \frac{\posterior(m_1 \mid y)}{\posterior(m_2 \mid y)}
   =
   \frac{b(y \mid m_1)}{b(y \mid m_2)}
   \cdot
   \frac{\prior(m_1)}{\prior(m_2)}
$$
is called the \emph{Bayes factor}
\begin{equation} \label{eq:factor}
   \frac{b(y \mid m_1)}{b(y \mid m_2)}
\end{equation}
the ratio of posterior odds to prior odds.

The prior odds tells how the prior compares the probability of the models.
The Bayes factor tells us how the data shifts that comparison going from
prior to posterior via Bayes rule.
Bayes factors are the primary tool Bayesians use for model comparison,
the competitor for frequentist $P$-values in frequentist hypothesis
tests of model comparison.

Note that our clumsy multiple letter notation for priors and posteriors
$\prior(m)$ and $\posterior(m \mid y)$ does not matter because neither
is involved in the actual calculation of Bayes factors \eqref{eq:factor}.
Priors and posteriors are involved in motivating Bayes factors but not in
calculating them.

\subsection{Tempering} \label{sec:temper}

Simulated tempering \citep{marinari-parisi,geyer-thompson} is a method of
Markov chain Monte Carlo (MCMC) simulation of many distributions at once.
It was originally invented with the primary aim of speeding up MCMC
convergence, but was also recognized to be useful for sampling multiple
distributions \citep{geyer-thompson}.  In the latter role it is sometimes
referred to as ``umbrella sampling'' which is a term coined
by \citet{torrie-valleau} for sampling multiple distributions not necessarily
via MCMC or simulated tempering (which was invented long after they wrote).

We have a finite set of unnormalized distributions we want to sample,
all related in some way.  The R function \texttt{temper}
in the CRAN package \texttt{mcmc}
requires all to have continuous distributions for random vectors of the same
dimension (all distributions have the same domain $\real^p$).
Let $h_i$, $i \in \mathcal{I}$ denote the unnormalized densities of
these distributions.  Simulated tempering (called ``serial tempering'' by
the \texttt{temper} function to distinguish from a related scheme not used
in this document called ``parallel tempering'' and in either case abbreviated
ST) runs a Markov chain whose
state is a pair $(i, x)$ where $i \in \mathcal{I}$ and $x \in \real^p$.

The unnormalized density of stationary distribution of the ST chain is
\begin{equation} \label{eq:st-joint}
   h(i, x) = h_i(x) c_i
\end{equation}
where the $c_i$ are arbitrary constants chosen by the user (more on this later).

The equilibrium distribution of the ST state $(I, X)$ --- both bits random ---
is such that conditional distribution of $X$ given $I = i$ is the distribution
with unnormalized density $h_i$.  This is obvious from $h(i, x)$ being the
unnormalized conditional density --- the same function thought of as
a function of both variables is the unnormalized joint density and thought
of as a function of just one of the variables is an unnormalized conditional
density --- and $h(i, x)$ thought of as a function of $x$ for fixed $i$ being
proportional to $h_i$.  The equilibrium unnormalized marginal distribution
of $I$ is
\begin{equation} \label{eq:margin}
   \int h(i, x) \, d x = c_i \int h_i(x) \, d x = c_i d_i
\end{equation}
where
$$
   d_i = \int h_i(x) \, d x
$$
is the normalizing constant for $h_i$, that is, $h_i / d_i$ is a normalized
distribution.

It is clear from \eqref{eq:margin} being the unnormalized marginal distribution
that in order for the marginal distribution to be uniform we must choose the
tuning constants $c_i$ to be proportional to $1 / d_i$.  It is not important
that the marginal distribution be exactly uniform, but unless it is
approximately uniform, the sampler will not visit each distribution frequently.
Thus we do need to have the $c_i$ to be approximately proportional to $1 / d_i$.
This is accomplished by trial and
error (one example is done in this document) and is easy for easy problems
and hard for hard problems \citep[have much to say about adjusting
the $c_i$]{geyer-thompson}.  For the rest of this section we will assume
the tuning constants $c_i$ have been so adjusted:
we do not have the $c_i$ exactly proportional to $1 / d_i$ but do have
them approximately proportional to $1 / d_i$.

\subsection{Tempering and Bayes Factors}

Bayes factors are very important in Bayesian inference and many methods have
been invented to calculate them.  No method except the one described here
using ST is anywhere near as accurate and straightforward.  Thus no competitors
will be discussed.

In using ST for Bayes factors we identify the index set $\mathcal{I}$ with
the model set $\mathcal{M}$ and use the integers 1, $\ldots$, $k$ for both.
We would like to identify the within model parameter vector $\theta$ with
the vector $x$ that is the continuous part of the state of the ST Markov
chain, but cannot because the dimension of $\theta$ depends on $m$ and this
is not allowed.  Thus we have to do something a bit more complicated.  We
``pad'' $\theta$ so that it always has the same dimension, doing so in
a way that does not interfere with the Bayes factor calculation.  Write
$\theta = (\theta_{\text{actual}}, \theta_{\text{pad}})$, the dimension
of both parts depending on the model $m$.  Then we insist on the following
conditions:
$$
   f(y \mid \theta, m) = f(y \mid \theta_{\text{actual}}, m)
$$
so the data distribution does not depend on the ``padding'' and
$$
   g(\theta \mid m) = g_{\text{actual}}(\theta_{\text{actual}} \mid m)
   \cdot g_{\text{pad}}(\theta_{\text{pad}} \mid m)
$$
so the two parts are \emph{a priori} independent and both parts of the prior
are normalized proper priors.  This assures that
\begin{equation} \label{eq:unnormalized-bayes-factors}
\begin{split}
   b(y \mid m)
   & =
   \int_{\Theta_m} f(y \mid \theta, m) g(\theta \mid m) \, d \theta
   \\
   & =
   \iint f(y \mid \theta_{\text{actual}}, m)
   g_{\text{actual}}(\theta_{\text{actual}} \mid m)
   g_{\text{pad}}(\theta_{\text{pad}} \mid m)
   \, d \theta_{\text{actual}}
   \, d \theta_{\text{pad}}
   \\
   & =
   \int_{\Theta_m} f(y \mid \theta_{\text{actual}}, m)
   g_{\text{actual}}(\theta_{\text{actual}} \mid m)
   \, d \theta_{\text{actual}}
\end{split}
\end{equation}
so the calculation of the unnormalized Bayes factors is the same whether
or not we ``pad'' $\theta$, and we may then take
\begin{align*}
   h_m(\theta)
   & = 
   f(y \mid \theta, m) g(\theta \mid m)
   \\
   & =
   f(y \mid \theta_{\text{actual}}, m)
   g_{\text{actual}}(\theta_{\text{actual}} \mid m)
   g_{\text{pad}}(\theta_{\text{pad}} \mid m)
\end{align*}
to be the unnormalized densities for the component distributions of the ST
chain, in which case the unnormalized Bayes factors are proportional to the
normalizing constants $d_i$ in Section~\ref{sec:temper}.

\subsection{Tempering and Normalizing Constants}

Let $d$ be the normalizing constant for the joint equilibrium distribution
of the ST chain \eqref{eq:st-joint}.  When we are running the ST chain we know
neither $d$ nor the $d_i$ but we do know the $c_i$, which are constants we
have chosen based on the results of previous runs but are fixed known numbers
for the current run.  Let $(I_t, X_t)$, $t = 1$, 2, $\ldots$ be the sample
path of the ST chain.  Recall that (somewhat annoyingly) we are using the
notation $(i, x)$ for the state vector of a general ST chain and the notation
$(m, \theta)$ for ST chains used to calculate Bayes factors, identifying
$i = m$ and $x = \theta$.

Let $\indicator(\fatdot)$ denote the function that maps logical values to
numerical values, false to zero and true to one.  Normalizing constants are
estimated by averaging the time spent in each model
\begin{equation} \label{eq:st-estimates}
   \hat{\delta}_n(m) = \frac{1}{n} \sum_{t = 1}^n \indicator(I_t = m)
\end{equation}
For the purposes of approximating Bayes factors the $X_t$ are ignored.
The $X_t$ may be useful for other purposes, such as
Bayesian model averaging \citep*{bma}, but this is not discussed here.

The Monte Carlo approximations \eqref{eq:st-estimates} converge
to their expected values under the equilibrium distribution
\begin{equation} \label{eq:st-expectations}
   E\{ \indicator(I_t = m) \}
   =
   \int \frac{h(m, x)}{d} \, d x
   =
   \frac{c_m d_m}{d}
   =
   \delta(m)
\end{equation}
We want to estimate the unnormalized Bayes factors
\eqref{eq:unnormalized-bayes-factors}, which are in this context proportional
to the $d_m$.  The $c_m$ are known, $d$ is unknown but does not matter since
we only need to estimate the $d_m = b(m \mid y)$ up to an overall unknown
constant of proportionality, which cancels out of Bayes factors
\eqref{eq:factor}.

Note that our discussion here applies unchanged to the general problem of
estimating normalizing constants up to an unknown constant of proportionality,
which has applications other than Bayes factors, for example, missing data
maximum likelihood \citep{thompson-guo,geyer,sung-geyer}.
The ST method approximates normalizing constants up to an overall constant of
proportionality with high accuracy regardless of how large or small they are
(whether they are $10^{100}$ or $10^{-100}$), and no other method that does
not use essentially the same idea can do this.

The key is what seems at first sight to be a weakness of ST, the need to
adjust the tuning constants $c_i$ by trial and error.  In this context the
weakness is actually a strength: the adjusted $c_i$ contain most of the
information about the size of the normalizing constants $d_i$ and the
Monte Carlo averages \eqref{eq:st-estimates} add only the finishing touch.
Thus multiple runs of the ST chain with different choices of the $c_i$ used
in each run are needed (the ``trial and error''), but the information from
all are incorporated in the final run used for final approximation of the
normalizing constants (Bayes factors).  It is perhaps surprising that the
Monte Carlo error approximation is trivial.  In the context of the last run
of the ST chain the $c_i$ are known constants and contribute no error.
The Monte Carlo error of the averages \eqref{eq:st-estimates} is
straightforwardly estimated by batch means or competing methods.

\citet{geyer-thompson} note that the $c_i$ enter formally like a prior:
one can think of $h_i(x) c_i$ as likelihood times prior.  But one should
not think of the $c_i$ as representing prior information, informative,
non-informative, or in between.  The $c_i$ are adjusted to make the ST
distribution sample all the models $h_i$, and that is the only criterion
for the adjustment.  For this reason \citet{geyer-thompson} call the
$c_i$ the \emph{pseudoprior}.  This is a special case of a general principle
of MCMC.  When doing MCMC one should forget the statistical motivation
(in this case Bayes factors).  One should set up a Markov chain that does
a good job of simulating the required equilibrium distribution, whatever
it is.  Thinking about the statistical motivation of the equilibrium does
not help and can hurt (if one thinks of the pseudoprior as an actual prior,
one may be tempted to adjust it to represent prior information).

\section{R Package MCMC}

We use the R statistical computing environment \citep{rcore} in our analysis.
It is free software and can be obtained from
\url{http://cran.r-project.org}.  Precompiled binaries
are available for Windows, Macintosh, and popular Linux distributions.
We use the contributed package \verb@mcmc@ \citep{mcmc-R-package}
If R has been installed, but this package has
not yet been installed, do
\begin{verbatim}
install.packages("mcmc")
\end{verbatim}
from the R command line
(or do the equivalent using the GUI menus if on Apple Macintosh
or Microsoft Windows).  This may require root or administrator privileges.

Assuming the \verb@mcmc@ package has been installed, we load it
\begin{Schunk}
\begin{Sinput}
> library(mcmc)
\end{Sinput}
\end{Schunk}
The version of the package used to make this document
is 0.7-3 (which is available on CRAN).
The version of R used to make this document is 2.10.1.

We also set the random number generator seed so that the results are
reproducible.
\begin{Schunk}
\begin{Sinput}
> set.seed(42)
\end{Sinput}
\end{Schunk}
To get different results, change the setting or don't set the seed at all.

\section{Logistic Regression Example}

We use the same logistic regression example used in the \texttt{mcmc}
package vignette for the \texttt{metrop} function (file \texttt{demo.pdf}.
Simulated data for the problem are in the data set \verb@logit@.
There are five variables in the data set, the response \verb@y@
and four predictors, \verb@x1@, \verb@x2@, \verb@x3@, and \verb@x4@.

A frequentist analysis for the problem is done by the following R statements
\begin{Schunk}
\begin{Sinput}
> data(logit)
> out <- glm(y ~ x1 + x2 + x3 + x4, data = logit, family = binomial, 
+     x = TRUE)
> summary(out)
\end{Sinput}
\begin{Soutput}
Call:
glm(formula = y ~ x1 + x2 + x3 + x4, family = binomial, data = logit, 
    x = TRUE)

Deviance Residuals: 
    Min       1Q   Median       3Q      Max  
-1.7461  -0.6907   0.1540   0.7041   2.1943  

Coefficients:
            Estimate Std. Error z value Pr(>|z|)   
(Intercept)   0.6328     0.3007   2.104  0.03536 * 
x1            0.7390     0.3616   2.043  0.04100 * 
x2            1.1137     0.3627   3.071  0.00213 **
x3            0.4781     0.3538   1.351  0.17663   
x4            0.6944     0.3989   1.741  0.08172 . 
---
Signif. codes:  0 ‘***’ 0.001 ‘**’ 0.01 ‘*’ 0.05 ‘.’ 0.1 ‘ ’ 1 

(Dispersion parameter for binomial family taken to be 1)

    Null deviance: 137.628  on 99  degrees of freedom
Residual deviance:  87.668  on 95  degrees of freedom
AIC: 97.668

Number of Fisher Scoring iterations: 6
\end{Soutput}
\end{Schunk}

But this example isn't about frequentist analysis, we want a Bayesian
analysis.  For our Bayesian analysis we assume the same data model as the
frequentist, and we assume the prior distribution of the five parameters
(the regression coefficients) makes them independent and identically
normally distributed with mean 0 and standard deviation 2.

Moreover, we wish to calculate Bayes factors for the $16 = 2^4$ possible
submodels that include or exclude each of the
predictors, \verb@x1@, \verb@x2@, \verb@x3@, and \verb@x4@.

\subsection{Setup}

We set up a matrix that indicates these models.
\begin{Schunk}
\begin{Sinput}
> varnam <- names(coefficients(out))
> varnam <- varnam[varnam != "(Intercept)"]
> nvar <- length(varnam)
> models <- NULL
> foo <- seq(0, 2^nvar - 1)
> for (i in 1:nvar) {
+     bar <- foo%/%2^(i - 1)
+     bar <- bar%%2
+     models <- cbind(bar, models, deparse.level = 0)
+ }
> colnames(models) <- varnam
> models
\end{Sinput}
\begin{Soutput}
      x1 x2 x3 x4
 [1,]  0  0  0  0
 [2,]  0  0  0  1
 [3,]  0  0  1  0
 [4,]  0  0  1  1
 [5,]  0  1  0  0
 [6,]  0  1  0  1
 [7,]  0  1  1  0
 [8,]  0  1  1  1
 [9,]  1  0  0  0
[10,]  1  0  0  1
[11,]  1  0  1  0
[12,]  1  0  1  1
[13,]  1  1  0  0
[14,]  1  1  0  1
[15,]  1  1  1  0
[16,]  1  1  1  1
\end{Soutput}
\end{Schunk}
In each row, 1 indicates the predictor is in the model and 0 indicates it is
out.

The function \texttt{temper} in the \text{mcmc} package that does tempering
requires a notion of neighbors among models.  It attempts jumps only between
neighboring models.  Here we choose models to be neighbors if they differ
only by one predictor.
\begin{Schunk}
\begin{Sinput}
> neighbors <- matrix(FALSE, nrow(models), nrow(models))
> for (i in 1:nrow(neighbors)) {
+     for (j in 1:ncol(neighbors)) {
+         foo <- models[i, ]
+         bar <- models[j, ]
+         if (sum(foo != bar) == 1) 
+             neighbors[i, j] <- TRUE
+     }
+ }
\end{Sinput}
\end{Schunk}

Now we specify the equilibrium distribution of the ST chain.  Its state vector
is $(i, x)$ or $(m, \theta)$ in our alternative notations, where $i$ is an
integer between $1$ and \verb@nrow(models)@ = 16 and
$\theta$ is the parameter vector ``padded'' to always be the same length,
so we take it to be the length of the parameter vector of the full model
which is \verb@length(out$coefficients)@ or \verb@ncol(models) + 1@ which makes
the length of the state of the ST chain \verb@ncol(models) + 2@.
We take the within model priors for the ``padded'' components of the parameter
vector to be the same as those for the ``actual'' components, normal with
mean 0 and standard deviation 2 for all cases.
As is seen in \eqref{eq:unnormalized-bayes-factors} the priors for the
``padded'' components (parameters not in the model for the current state)
do not matter because they drop out of the Bayes factor calculation.
The choice does not matter much for this toy example.
See the discussion section for more on this issue.
It is important that we use normalized log priors,
the term \verb@dnorm(beta, 0, 2, log = TRUE)@ in the function, unlike
when we are simulating only one model as in the \texttt{mcmc} package vignette
where it would be o.~k.\ to use unnormalized log priors \verb@- beta^2 / 8@.
The \texttt{temper} function wants the log unnormalized density of the
equilibrium distribution.
We include an additional argument \texttt{log.pseudo.prior},
which is $\log(c_i)$ in our mathematical development, because this changes
from run to run as we adjust it by trial and error.  Other ``arguments''
are the model matrix of the full model \texttt{modmat}, the matrix
\texttt{models} relating integer indices (the first component of the state
vector of the ST chain) to which predictors are in or out of the model,
and the data vector \texttt{y}, but these are not passed as arguments to our
function and instead are found in the R global environment.
\begin{Schunk}
\begin{Sinput}
> modmat <- out$x
> y <- logit$y
> ludfun <- function(state, log.pseudo.prior) {
+     stopifnot(is.numeric(state))
+     stopifnot(length(state) == ncol(models) + 2)
+     icomp <- state[1]
+     stopifnot(icomp == as.integer(icomp))
+     stopifnot(1 <= icomp && icomp <= nrow(models))
+     stopifnot(is.numeric(log.pseudo.prior))
+     stopifnot(length(log.pseudo.prior) == nrow(models))
+     beta <- state[-1]
+     inies <- c(TRUE, as.logical(models[icomp, ]))
+     beta.logl <- beta
+     beta.logl[!inies] <- 0
+     eta <- as.numeric(modmat %*% beta.logl)
+     logp <- ifelse(eta < 0, eta - log1p(exp(eta)), 
+         -log1p(exp(-eta)))
+     logq <- ifelse(eta < 0, -log1p(exp(eta)), -eta - 
+         log1p(exp(-eta)))
+     logl <- sum(logp[y == 1]) + sum(logq[y == 0])
+     logl + sum(dnorm(beta, 0, 2, log = TRUE)) + log.pseudo.prior[icomp]
+ }
\end{Sinput}
\end{Schunk}

\subsection{Trial and Error}

Now we are ready to try it out.  We start in the full model at its MLE,
and we initialize \texttt{log.pseudo.prior} at all zeros, having no idea
\emph{a priori} what it should be.
\begin{Schunk}
\begin{Sinput}
> state.initial <- c(nrow(models), out$coefficients)
> qux <- rep(0, nrow(models))
> out <- temper(ludfun, initial = state.initial, neighbors = neighbors, 
+     nbatch = 1000, blen = 100, log.pseudo.prior = qux)
> names(out)
\end{Sinput}
\begin{Soutput}
 [1] "lud"          "initial"      "neighbors"    "nbatch"      
 [5] "blen"         "nspac"        "scale"        "outfun"      
 [9] "debug"        "parallel"     "initial.seed" "final.seed"  
[13] "time"         "batch"        "acceptx"      "accepti"     
[17] "initial"      "final"        "ibatch"      
\end{Soutput}
\begin{Sinput}
> out$time
\end{Sinput}
\begin{Soutput}
   user  system elapsed 
 27.357   0.000  27.356 
\end{Soutput}
\end{Schunk}
So what happened?
\begin{Schunk}
\begin{Sinput}
> ibar <- colMeans(out$ibatch)
> ibar
\end{Sinput}
\begin{Soutput}
 [1] 0.00000 0.00000 0.00000 0.00000 0.00524 0.06489 0.00754
 [8] 0.06021 0.00033 0.00202 0.00008 0.00054 0.28473 0.31487
[15] 0.12478 0.13477
\end{Soutput}
\end{Schunk}
The ST chain did not mix well, several models not being visited even once.
So we adjust the pseudo priors to get uniform distribution.
\begin{Schunk}
\begin{Sinput}
> qux <- qux + pmin(log(max(ibar)/ibar), 10)
> qux <- qux - min(qux)
> qux
\end{Sinput}
\begin{Soutput}
 [1] 10.0000000 10.0000000 10.0000000 10.0000000  4.0958384
 [6]  1.5794663  3.7319377  1.6543214  6.8608225  5.0490623
[11]  8.2778885  6.3683460  0.1006185  0.0000000  0.9256077
[16]  0.8485902
\end{Soutput}
\end{Schunk}
The new pseudoprior should be proportional to \verb@1 / ibar@ if \texttt{ibar}
is an accurate estimate of \eqref{eq:st-expectations}, but this makes no sense
when the estimates are bad, in particular, when the are exactly zero.  Thus
we put an upper bound, chosen arbitrarily (here 10) on the maximum increase
of the log pseudoprior.  The statement
\begin{verbatim}
qux <- qux - min(qux)
\end{verbatim}
is unnecessary.  An overall arbitrary constant can be added to
the log pseudoprior without changing the equilibrium distribution of the
ST chain.
We do this only to make \texttt{qux} more comparable from
run to run.

Now we repeat this until the log pseudoprior ``converges'' roughly.
\begin{Schunk}
\begin{Sinput}
> qux.save <- qux
> time.save <- out$time
> repeat {
+     out <- temper(out, log.pseudo.prior = qux)
+     ibar <- colMeans(out$ibatch)
+     qux <- qux + pmin(log(max(ibar)/ibar), 10)
+     qux <- qux - min(qux)
+     qux.save <- rbind(qux.save, qux, deparse.level = 0)
+     time.save <- rbind(time.save, out$time, deparse.level = 0)
+     if (max(ibar)/min(ibar) < 2) 
+         break
+ }
> qux.save
\end{Sinput}
\begin{Soutput}
         [,1]      [,2]     [,3]      [,4]     [,5]     [,6]
[1,] 10.00000 10.000000 10.00000 10.000000 4.095838 1.579466
[2,] 17.70751  9.999775 14.43037  9.714906 4.049512 1.539382
[3,] 18.76818  9.325494 14.43382  9.014132 3.972982 1.385058
[4,] 18.94703  9.733071 14.71371  9.478451 4.276229 1.490810
         [,7]     [,8]     [,9]    [,10]    [,11]    [,12]
[1,] 3.731938 1.654321 6.860822 5.049062 8.277888 6.368346
[2,] 3.349201 1.826804 6.024402 4.412902 6.607937 5.676098
[3,] 3.164068 1.401334 5.635094 4.145408 6.330649 5.173185
[4,] 3.229581 1.553204 6.186982 4.623995 6.881493 5.660156
         [,13]       [,14]     [,15]     [,16]
[1,] 0.1006185 0.000000000 0.9256077 0.8485902
[2,] 0.1475452 0.000000000 0.7369393 0.7865126
[3,] 0.0000000 0.003240203 0.4411160 0.6143926
[4,] 0.1095034 0.000000000 0.9146666 0.8328578
\end{Soutput}
\begin{Sinput}
> qux
\end{Sinput}
\begin{Soutput}
 [1] 18.9470263  9.7330707 14.7137106  9.4784515  4.2762293
 [6]  1.4908095  3.2295806  1.5532045  6.1869824  4.6239946
[11]  6.8814927  5.6601564  0.1095034  0.0000000  0.9146666
[16]  0.8328578
\end{Soutput}
\begin{Sinput}
> apply(time.save, 2, sum)
\end{Sinput}
\begin{Soutput}
 user.self   sys.self    elapsed user.child  sys.child 
   108.474      0.000    108.464      0.000      0.000 
\end{Soutput}
\end{Schunk}

Now that the pseudoprior is adjusted well enough, we need to perhaps
make other adjustments to get acceptance rates near 20\%.
\begin{Schunk}
\begin{Sinput}
> out$accepti
\end{Sinput}
\begin{Soutput}
           [,1]      [,2]      [,3]      [,4]      [,5]
 [1,]        NA 0.2392567 0.2122222        NA 0.2207637
 [2,] 0.2941989        NA        NA 0.2500000        NA
 [3,] 0.2362003        NA        NA 0.2609254        NA
 [4,]        NA 0.2775120 0.3292496        NA        NA
 [5,] 0.2199747        NA        NA        NA        NA
 [6,]        NA 0.1552463        NA        NA 0.2347826
 [7,]        NA        NA 0.1579487        NA 0.2902893
 [8,]        NA        NA        NA 0.1472192        NA
 [9,] 0.3443478        NA        NA        NA        NA
[10,]        NA 0.2300469        NA        NA        NA
[11,]        NA        NA 0.2327448        NA        NA
[12,]        NA        NA        NA 0.2503864        NA
[13,]        NA        NA        NA        NA 0.2430998
[14,]        NA        NA        NA        NA        NA
[15,]        NA        NA        NA        NA        NA
[16,]        NA        NA        NA        NA        NA
           [,6]      [,7]      [,8]      [,9]     [,10]
 [1,]        NA        NA        NA 0.2182030        NA
 [2,] 0.1925287        NA        NA        NA 0.2299383
 [3,]        NA 0.1992574        NA        NA        NA
 [4,]        NA        NA 0.2157191        NA        NA
 [5,] 0.2871537 0.3480454        NA        NA        NA
 [6,]        NA        NA 0.2943005        NA        NA
 [7,]        NA        NA 0.2502629        NA        NA
 [8,] 0.3119266 0.2540193        NA        NA        NA
 [9,]        NA        NA        NA        NA 0.2508651
[10,]        NA        NA        NA 0.2416107        NA
[11,]        NA        NA        NA 0.2125436        NA
[12,]        NA        NA        NA        NA 0.2960630
[13,]        NA        NA        NA 0.1707060        NA
[14,] 0.2347140        NA        NA        NA 0.1394892
[15,]        NA 0.3637902        NA        NA        NA
[16,]        NA        NA 0.2719072        NA        NA
          [,11]     [,12]     [,13]     [,14]     [,15]
 [1,]        NA        NA        NA        NA        NA
 [2,]        NA        NA        NA        NA        NA
 [3,] 0.2065789        NA        NA        NA        NA
 [4,]        NA 0.2503937        NA        NA        NA
 [5,]        NA        NA 0.3187415        NA        NA
 [6,]        NA        NA        NA 0.2373950        NA
 [7,]        NA        NA        NA        NA 0.2285714
 [8,]        NA        NA        NA        NA        NA
 [9,] 0.1849427        NA 0.2535885        NA        NA
[10,]        NA 0.2768116        NA 0.2201550        NA
[11,]        NA 0.2859451        NA        NA 0.2173228
[12,] 0.2787402        NA        NA        NA        NA
[13,]        NA        NA        NA 0.3325688 0.2688889
[14,]        NA        NA 0.2721154        NA        NA
[15,] 0.2166934        NA 0.3615160        NA        NA
[16,]        NA 0.1775586        NA 0.3317536 0.2301496
          [,16]
 [1,]        NA
 [2,]        NA
 [3,]        NA
 [4,]        NA
 [5,]        NA
 [6,]        NA
 [7,]        NA
 [8,] 0.2508631
 [9,]        NA
[10,]        NA
[11,]        NA
[12,] 0.2364532
[13,]        NA
[14,] 0.2659574
[15,] 0.2916053
[16,]        NA
\end{Soutput}
\begin{Sinput}
> out$acceptx
\end{Sinput}
\begin{Soutput}
 [1] 0.19633658 0.09694989 0.07855559 0.05680561 0.10568032
 [6] 0.05587045 0.05044360 0.03606650 0.10563674 0.06666667
[11] 0.05854701 0.03254552 0.06180666 0.03780718 0.03388525
[16] 0.02426036
\end{Soutput}
\end{Schunk}
The acceptance rates for swaps seem o. k.
\begin{Schunk}
\begin{Sinput}
> min(as.vector(out$accepti), na.rm = TRUE)
\end{Sinput}
\begin{Soutput}
[1] 0.1394892
\end{Soutput}
\end{Schunk}
and there is nothing simple we can do to adjust them (adjustment is possible,
see the discussion section for more on this issue).  We adjust the
acceptance rates for within model moves by adjusting the scaling.
\begin{Schunk}
\begin{Sinput}
> out <- temper(out, scale = 0.5, log.pseudo.prior = qux)
> time.save <- rbind(time.save, out$time, deparse.level = 0)
> out$acceptx
\end{Sinput}
\begin{Soutput}
 [1] 0.4003067 0.2870911 0.2511125 0.2217288 0.3143193 0.2468896
 [7] 0.2245908 0.2063492 0.3088322 0.2260388 0.2026928 0.1677290
[13] 0.2470217 0.1932432 0.1964912 0.1526632
\end{Soutput}
\end{Schunk}
Looks o.~k.\ now.

Inspection of autocorrelation functions for components
of \verb@out$ibatch@ (not shown) says batch length needs to be at least
4 times longer.  We make it 10 times longer for safety.
\begin{Schunk}
\begin{Sinput}
> out <- temper(out, blen = 10 * out$blen, log.pseudo.prior = qux)
> time.save <- rbind(time.save, out$time, deparse.level = 0)
> foo <- apply(time.save, 2, sum)
> foo.min <- floor(foo[1]/60)
> foo.sec <- foo[1] - 60 * foo.min
> c(foo.min, foo.sec)
\end{Sinput}
\begin{Soutput}
user.self user.self 
    6.000    46.473 
\end{Soutput}
\end{Schunk}
The total time for all runs of the temper function was
6 minutes and 46.5 seconds.

\subsection{Bayes Factor Calculations}

Now we calculate log 10 Bayes factors relative to the model with the highest
unnormalized Bayes factor.
\begin{Schunk}
\begin{Sinput}
> log.10.unnorm.bayes <- (qux - log(colMeans(out$ibatch)))/log(10)
> k <- seq(along = log.10.unnorm.bayes)[log.10.unnorm.bayes == 
+     min(log.10.unnorm.bayes)]
> models[k, ]
\end{Sinput}
\begin{Soutput}
x1 x2 x3 x4 
 1  1  0  1 
\end{Soutput}
\begin{Sinput}
> log.10.bayes <- log.10.unnorm.bayes - log.10.unnorm.bayes[k]
> log.10.bayes
\end{Sinput}
\begin{Soutput}
 [1] 8.17814103 4.17098637 6.33069128 4.05292216 1.80254545
 [6] 0.67203156 1.40468558 0.70498671 2.58875400 1.93202268
[11] 2.82341431 2.37170521 0.08004553 0.00000000 0.37357715
[16] 0.35242443
\end{Soutput}
\end{Schunk}
These are base 10 logarithms of the Bayes factors against the $k$-th
model where $k = 14$.  For example, the Bayes factor for the $k$-th
model divided by the Bayes factor for the first model is
$10^{8.178}$.

Now we calculate Monte Carlo standard errors two different ways.  One is
the way the delta method is usually taught.  To simplify notation, denote
the Bayes factors
$$
   b_m = b(y \mid m)
$$
and their Monte Carlo approximations $\hat{b}_m$.  Then the log Bayes factors
are
$$
   g_i(b) = \log_{10} b_i - \log_{10} b_k
$$
hence we need to apply the delta method with the function $g_i$, which has
derivatives
\begin{align*}
   \frac{\partial g_i(b)}{\partial b_i}
   & =
   \frac{1}{b_i \log_e(10)}
   \\
   \frac{\partial g_i(b)}{\partial b_k}
   & =
   - \frac{1}{b_k \log_e(10)}
   \\
   \frac{\partial g_i(b)}{\partial b_j}
   & =
   0, \qquad \text{$j \neq i$ and $j \neq k$}
\end{align*}
\begin{Schunk}
\begin{Sinput}
> fred <- var(out$ibatch)/out$nbatch
> sally <- colMeans(out$ibatch)
> mcse.log.10.bayes <- (1/log(10)) * sqrt(diag(fred)/sally^2 - 
+     2 * fred[, k]/(sally * sally[k]) + fred[k, k]/sally[k]^2)
> mcse.log.10.bayes
\end{Sinput}
\begin{Soutput}
 [1] 0.02297789 0.02297554 0.02427784 0.02466045 0.02124325
 [6] 0.01925444 0.02381757 0.02289809 0.02190834 0.01983693
[11] 0.02293685 0.02213382 0.01753144 0.00000000 0.02146088
[16] 0.01857454
\end{Soutput}
\begin{Sinput}
> foompter <- cbind(models, log.10.bayes, mcse.log.10.bayes)
> round(foompter, 5)
\end{Sinput}
\begin{Soutput}
      x1 x2 x3 x4 log.10.bayes mcse.log.10.bayes
 [1,]  0  0  0  0      8.17814           0.02298
 [2,]  0  0  0  1      4.17099           0.02298
 [3,]  0  0  1  0      6.33069           0.02428
 [4,]  0  0  1  1      4.05292           0.02466
 [5,]  0  1  0  0      1.80255           0.02124
 [6,]  0  1  0  1      0.67203           0.01925
 [7,]  0  1  1  0      1.40469           0.02382
 [8,]  0  1  1  1      0.70499           0.02290
 [9,]  1  0  0  0      2.58875           0.02191
[10,]  1  0  0  1      1.93202           0.01984
[11,]  1  0  1  0      2.82341           0.02294
[12,]  1  0  1  1      2.37171           0.02213
[13,]  1  1  0  0      0.08005           0.01753
[14,]  1  1  0  1      0.00000           0.00000
[15,]  1  1  1  0      0.37358           0.02146
[16,]  1  1  1  1      0.35242           0.01857
\end{Soutput}
\end{Schunk}

An alternative calculation of the MCSE replaces the actual function
of the raw Bayes factors with its best linear approximation
$$
   \frac{1}{\log_e(10)} \left(\frac{\hat{b}_i - b_i}{b_i}
   - \frac{\hat{b}_k - b_k}{b_k} \right)
$$
and calculates the standard deviation of this quantity by batch means
\begin{Schunk}
\begin{Sinput}
> ibar <- colMeans(out$ibatch)
> herman <- sweep(out$ibatch, 2, ibar, "/")
> herman <- sweep(herman, 1, herman[, k], "-")
> mcse.log.10.bayes.too <- (1/log(10)) * apply(herman, 
+     2, sd)/sqrt(out$nbatch)
> all.equal(mcse.log.10.bayes, mcse.log.10.bayes.too)
\end{Sinput}
\begin{Soutput}
[1] TRUE
\end{Soutput}
\end{Schunk}

\section{Discussion}

We hope readers are impressed with the power of this method.  The key
to the method is pseudopriors adjusted by trial and error.  The method
could have been invented by any Bayesian who realized that the priors
on models, $\prior(m)$ in our notation in Section~\ref{sec:bayes-factors},
do not affect the Bayes factors and hence are irrelevant to calculating
Bayes factors.  Thus the priors (or pseudopriors in our terminology) should
be chosen for reasons of computational convenience, as we have done,
rather than to incorporate prior information.

The rest of the details of the method are unimportant.  The \texttt{temper}
function in R is convenient to use for this purpose, but there is no reason
to believe that it provides optimal sampling.  Samplers carefully designed
for each particular application would undoubtedly do better.  Our notion
of ``padding'' so that the within model parameters have the same dimension
for all models follows \citet{carlin-chib} but ``reversible jump'' samplers
\citep{green} would undoubtedly do better.  Unfortunately, there seems to
be no way to code up a function like \texttt{temper} that uses ``reversible
jump'' and requires no theoretical work from users that if messed up destroys
the algorithm.  The \texttt{temper} function is foolproof in the sense that
if the log unnormalized density function written by the user
(like our \texttt{ludfun}) is correct, then the ST Markov chain has the
equilibrium distribution is supposed to have.  There is nothing the
user can mess up except this user written function.  No analog of this
for ``reversible jump'' chains is apparent (to your humble author).

Two issues remain where the text above said ``see the discussion section for
more on this issue.''  The first was about within model priors for the
``padding'' components of within model parameter vectors
$g_{\text{pad}}(\theta_{\text{pad}} \mid m)$ in
the notation in \eqref{eq:unnormalized-bayes-factors}.
Rather that choose these so that they do not depend on the data (as we did),
it would be better (if more trouble) to choose them differently for each
``padding'' component, centering $g_{\text{pad}}(\theta_{\text{pad}} \mid m)$
so the distribution of a component of $\theta_{\text{pad}}$ is near to the
marginal distribution of the same component in neighboring models (according to
the \texttt{neighbors} argument of the \texttt{temper} function).

The other remaining issue is adjusting acceptance rates for jumps.  There
is no way to adjust this other than by changing the number of models and
their definitions.  But the models we have cannot be changed; if we are
to calculate Bayes factors form them, then we must sample them as they are.
But we can insert new models between old models.  For example,
if the acceptance for swaps between model $i$ and model $j$ is too low, then
we can insert distribution $k$ between them that has unnormalized density
$$
   h_k(x) = \sqrt{h_i(x) h_j(x)}.
$$
This idea is inherited from simulated tempering; \citep{geyer-thompson}
have much
discussion of how to insert additional distributions into a tempering network.
It is another key issue in using tempering to speed up sampling.  It is
less obvious in the Bayes factor context, but still an available technique
if needed.


\begin{thebibliography}{}

\bibitem[Carlin and Chib(1995)]{carlin-chib}
Carlin, B.~P. and Chib, S. (1995).
\newblock Bayesian model choice via Markov chain Monte Carlo methods.
\newblock \emph{Journal of the Royal Statistical Society, Series B},
    \textbf{57}, 473--484.

\bibitem[Geyer(1994)]{geyer}
Geyer, C.~J. (1994).
\newblock On the convergence of Monte Carlo maximum likelihood calculations.
\newblock \emph{Journal of the Royal Statistical Society, Series B},
    \textbf{56} 261--274.

\bibitem[Geyer., 2009]{mcmc-R-package}
Geyer., C.~J. (2009).
\newblock \emph{mcmc: Markov Chain Monte Carlo}.
\newblock R package version 0.7-2, available from CRAN.

\bibitem[Geyer and Thompson(1995)]{geyer-thompson}
Geyer, C.~J., and Thompson, E.~A. (1995).
\newblock Annealing Markov chain Monte Carlo with applications to ancestral
    inference.
\newblock \emph{Journal of the American Statistical Association}, \textbf{90},
    909--920.

\bibitem[Green(1995)]{green}
Green, P.~J. (1995).
\newblock Reversible jump {M}arkov chain {M}onte {C}arlo computation and
  {B}ayesian model determination.
\newblock \emph{Biometrika}, \textbf{82}, 711--732.

\bibitem[Hoeting et al.(1999)Hoeting, Madigan, Raftery, and Volinsky]{bma}
Hoeting, J.~A., Madigan, D., Raftery, A.~E. and Volinsky, C.~T. (1999).
\newblock Bayesian model averaging: A tutorial (with discussion).
\newblock \emph{Statical Science}, \textbf{19}, 382--417.
\newblock The version printed in the journal had the equations messed up in
    the production process; a corrected version is available at
    \url{http://www.stat.washington.edu/www/research/online/1999/hoeting.pdf}.

\bibitem[Marinari and Parisi(1992)]{marinari-parisi}
Marinari, E., and Parisi G. (1992).
\newblock Simulated tempering: A new Monte Carlo Scheme.
\newblock \emph{Europhysics Letters}, \textbf{19}, 451--458.

\bibitem[R Development Core Team(2010)]{rcore}
R Development Core Team (2010).
\newblock R: A language and environment for statistical computing.
\newblock R Foundation for Statistical Computing, Vienna, Austria.
\newblock \url{http://www.R-project.org}.

\bibitem[Sung and Geyer(2007)]{sung-geyer}
Sung, Y.~J. and Geyer, C.~J. (2007).
\newblock Monte Carlo likelihood inference for missing data models.
\newblock \emph{Annals of Statistics}, \textbf{35}, 990--1011.

\bibitem[Thompson and Guo(1991)]{thompson-guo}
Thompson, E. A. and Guo, S. W. (1991).
\newblock Evaluation of likelihood ratios for complex genetic models.
\newblock \emph{IMA J. Math. Appl. Med. Biol.}, \textbf{8}, 149--169.

\bibitem[Torrie and Valleau(1977)]{torrie-valleau}
Torrie, G.~M., and Valleau, J.~P. (1977).
\newblock Nonphysical sampling distributions in Monte Carlo free-energy
  estimation: Umbrella sampling.
\newblock \emph{Journal of Computational Physics}, \textbf{23}, 187--199.

\end{thebibliography}

\end{document}



